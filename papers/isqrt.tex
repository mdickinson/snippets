\documentclass[a4paper]{article}

\usepackage{listings}
\usepackage{amsthm}
\usepackage{mathtools}

\DeclarePairedDelimiter\floor{\lfloor}{\rfloor}
\DeclarePairedDelimiter\ceil{\lceil}{\rceil}
\DeclarePairedDelimiter\abs{\lvert}{\rvert}
\DeclareMathOperator{\length}{length}

\theoremstyle{plain}
\newtheorem{lemma}{Lemma}

\theoremstyle{definition}
\newtheorem{definition}{Definition}
\newtheorem{example}{Example}

\title{Python's integer square root algorithm}
\author{Mark Dickinson}

\begin{document}
\lstset{language=Python}
\maketitle
\begin{abstract}
TBD
\end{abstract}
\section{Introduction}

In this paper we present a simple algorithm to compute the integer square root
of a nonnegative number. This algorithm has been implemented for Python's
\lstinline{math} module and is available in Python 3.8 as
\lstinline{math.isqrt}.

\section{Definitions}

\begin{definition}
  For a nonnegative integer $n$, the \emph{integer square root} of $n$ is
  the unique nonnegative integer $a$ satifying $a^2 \le n < (a + 1)^2$.
\end{definition}

To describe Python's algorithm for computing integer square roots, we introduce
a slightly weaker notion: that of a "near square root".

\begin{definition}
  Suppose $n$ is a positive integer. Call a positive integer $a$ a
  \emph{near square root} of $n$ if $(a - 1)^2 < n < (a + 1)^2$.
\end{definition}

Equivalently, $a$ is a near square root of $n$ if $a$ is either $\floor{\sqrt
n}$ or $\ceil{\sqrt n}$. In particular, if $n = a^2$ is a perfect square then
the only near square root of $n$ is $a$.

Given an algorithm to compute near square roots, it's straightforward to
compute integer square roots: suppose that $n$ is a positive integer and $a$ is
a near square root of $n$. If $a^2 \le n$ then $a^2 \le n < (a + 1)^2$ and so
$a$ is the integer square root of $n$. If not, then $(a - 1)^2 < n < a^2$ and
so $a - 1$ is the integer square root of $n$. Translating this into code,
and assuming that \lstinline{nsqrt} is a function that accepts a positive
integer and returns a near square root for that integer, we have:

\lstinputlisting[firstline=9,lastline=16]{isqrt_recursive.py}

\section{Newton's method}

In this section we describe a well-known approach to finding integer square
roots, based on a pure integer arithmetic version of Newton's method.

Fix a positive integer $n$. Define a function $f$ on the positive integers
by
$$f(a) = \floor*{\frac{a + n/a}2}.$$
Alternatively, we can define $f$ by
$$f(a) = \floor*{\frac{a + \floor{n/a}}2}$$
to make it clear that $f$ can be computed entirely within the domain of the
integers. The two forms are exactly equivalent (proof left to the reader), but
the first form is slightly more convenient for the proofs below.

In general, the function $f$ transforms an approximation to the integer square
root of $n$ into a better approximation. The following two lemmas make this
precise, and provide the foundation for using $f$ to compute integer square
roots.

\begin{lemma}
  For any positive integer $a$, $f(a)$ is greater than or equal to the
  integer square root of $n$.
\end{lemma}

\begin{proof}
  The AM-GM inequality applied to $a$ and $n/a$ gives
  $$\sqrt n \le \frac{a + n/a}2.$$
  Taking the floor of both sides gives the result.
\end{proof}

\begin{lemma}
  If a positive integer $a$ is strictly greater than the integer square
  root of $n$, $f(a) < a$.
\end{lemma}

\begin{proof}
  We have the following chain of equivalences:
  \begin{align}
    \floor{\sqrt n} < a &\iff \sqrt n < a \\
                        &\iff n < a^2 \\
                        &\iff n/a < a \\
                        &\iff a + n/a < 2a \\
                        &\iff \frac{a + n/a}2 < a \\
                        &\iff \floor*{\frac{a + n/a}2} < a \\
                        &\iff f(a) < a
  \end{align}
\end{proof}

Now suppose that we're given a starting guess $a$ for the integer square root
of $n$ that satisfies $\floor{\sqrt n} < a$. Then combining the lemmas, we
have
$$\floor{\sqrt n} \le f(a) < a.$$
If $f(a)$ is again strictly greater than the integer square root of $n$, we
further have:
$$\floor{\sqrt n} \le f(f(a)) < f(a) < a$$
and so on. Applying $f$ repeatedly to our current best guess creates a strictly
descreasing sequence of integers, bounded below by the integer square root of
$n$. That sequence must therefore reach the integer square root of $n$ in a
finite number of steps. The second lemma above also provides us with a
termination test: if $f(a) \ge a$ then $\floor{\sqrt n} \ge a$. If we also know
that $\floor{\sqrt n} \le a$, for example because $a$ is itself the result
of applying $f$ to some previous iterate, then $a$ must be the desired
integer square root.

The following code implements this idea, taking as starting guess the
smallest power of two that exceeds $\sqrt n$. The termination condition
$f(a) \ge a$ is replaced by the equivalent condition $\floor{n/a} \ge a$.

\lstinputlisting[lastline=10]{isqrt_newton.py}

\begin{example}
  Take $n = 16785408$, which is just a little larger than $2^{24} = 16777216$.
  Our starting guess for the square root is $2^{13} = 8192$. Successive
  iterations give $5120$, $4199$, $4098$, $4097$, and finally $4096$, which
  is the integer square root. Each iteration requires one division, and
  a final division is needed to establish the termination condition, for
  a total of six divisions.
\end{example}


\section{Computing near square roots recursively}

Newton's method provides a recursive approach for computing near square roots.
Suppose that $n$ is a positive integer, and that we wish to find a near square
root of $n$. Given a near square root $d$ of $\floor{n / k^2}$ for some
positive integer $k$, $dk$ is then an approximation to $\sqrt n$, although
possibly not a very good one. A single iteration of Newton's method applied to
$dk$ gives an improved approximation. Now comes the key point: if we're careful
not to choose $k$ too large with respect to $n$, we can prove that this
improved approximation will again be a near square root of $n$.

The following lemma makes this precise. Note that to keep calculations within
the domain of the integers, it's convenient to restrict $k$ to be even. So in
the lemma below, $k$ is replaced with $2m$.

\begin{lemma}
  Suppose that $n \ge 4$, and choose a positive integer $m$
  satisfying $4m^4 \le n$. Given a near square root $d$ of $\floor{n / 4m^2}$,
  define $a$ by
  $$ a = md + \floor*{\frac{n}{4md}}. $$
  Then $a$
  is a near square root of $n$.
\end{lemma}

\begin{proof}
  By definition of near square root, we have
  $$ (d - 1)^2 < \floor*{\frac{n}{4m^2}} < (d + 1)^2.$$
  Since $(d + 1)^2$ is an integer, we can remove the floor brackets to obtain
  $$ (d - 1)^2 < \frac{n}{4m^2} < (d + 1)^2.$$
  Taking square roots throughout and multiplying by $2m$ gives
  $$ 2m(d - 1) < \sqrt n < 2m(d + 1)$$
  which can be rearranged as
  $$ \abs{2md - \sqrt n} < 2m. $$
  Squaring and dividing through by $4md$ gives
  $$ 0 \le md + \frac{n}{4md} - \sqrt n < \frac md,$$
  which implies that
  $$ -1 < md + \floor*{\frac{n}{4md}} - \sqrt n < \frac{m}{d},$$
  Substituting the definition of $a$ gives
  $$ -1 < a - \sqrt n < m / d.$$
  To complete the proof, we need to know that $m \le d$, and so $m / d \le 1$.
  Indeed, from the assumption that $4m^4 \le n$ we have $m^2 \le n / 4m^2 < (d
  + 1)^2$, so $m < d + 1$. So now
  $$ -1 < a - \sqrt n < 1 $$
  from which $(a - 1)^2 < n < (a + 1)^2$, as required.
\end{proof}

\section{Implementation}

Like many arbitrary-precision integer implementations, Python's integer
implementation is based on binary, so multiplications and divisions by powers
of two can be performed efficiently via shifting. So for maximum efficiency when
applying the lemma above, we should choose our $m$ to be the largest power of
two satisfying $4m^4 \le n$. It's convenient to introduce another definition.

\begin{definition}
  For a nonnegative integer $n$, the \emph{bit length} of $n$, written
  $\length(n)$, is the least nonnegative integer $e$ for which $n < 2^e$.
\end{definition}

Equivalently, for positive $n$ the bit length of $n$ is $1 +
\floor{\log_2(n)}$, while the bit length of $0$ is $0$. For nonnegative
integers $n$ and $e$, we have $2^e \le n$ if and only if $e < \length(n)$.

When $m = 2^e$ is a power of two, the condition $4m^4 \le n$ of the lemma becomes $2^{2 +
4e} \le n$. That's equivalent to $2 + 4e < \length(n)$, or $e \le (\length(n) -
3) / 4$. Taking $e = \floor{(\length(n) - 3) / 4}$ gives the following
recursive near square root implementation, where the multiplication by $m$ and
the divisions by $4m$ and $4m^2$ are replaced by the corresponding bit-shift
operations.

\lstinputlisting[lastline=6]{isqrt_recursive.py}

\section{Iterative version}

We can unwind the recursion in the usual way to obtain the following
iterative version of the same algorithm, which is closer to the version
implemented in Python 3.8.

\lstinputlisting[lastline=14]{isqrt_iterative.py}

Here the quantities $c$, $d$, $e$, $k$ and $m$ are all small (machine-size)
integers. Only $a$ and $n$ are arbitrary-precision integers.

The number of iterations is exactly $\floor{log_2(\floor{log_2(n)})}$.


\end{document}
